\documentclass{article}
\usepackage{amsmath}
\usepackage{amsthm}
\usepackage{geometry}
% \usepackage{asm}

\newgeometry{
    top=0.7in,
    bottom=0.7in,
    outer=0.7in,
    inner=0.7in,
}

\theoremstyle{definition}
\newtheorem{definition}{Definition}[section]

\newtheorem{lem}{Lemma}
\newtheorem{clm}{Claim}
\newtheorem*{rem}{Remark}
\newtheorem{thm}{Theorem}
\newtheorem*{cor}{Corollary}


\begin{document}

\section{Intro}

We start with discribing the setting of game.
Given some sarch problem $\mathcal{S}$ on graphs i.e. $\mathcal{S}\subseteq \mathcal{G}_n \times O $
we define the two paraty sarch problem $\mathcal{S} \circ \text{or}_2$, that is each of players get some edages in the graph
and they need to solve togther the search problem over the union of thier graphs.

We will construct a famaly of graph gadgets such that solving $\mathcal{S} \circ \text{or}_2$ over graph with those gadgets insted of edages
is easy only if solving $\mathcal{S} \circ \text{IP}^{\log{n}}$ is easy. 
By lifting theorems we know that $\mathcal{S} \circ \text{IP}^{\log{n}}$ is as hard 
as the query complexity of $\mathcal{S}$ times factor of $\log{n}$.  

\subsection{Summary of Results}

\section{Constructions}

% \begin{definition}
Let $g_0$ be the following tree gadget of size $\log{n}$ that come as a replacment for an edge.
assume we have vertices $v,u$ in the orginal graph then we will have $v',u'$ in the new graph. 
Assume addionaly we have strings $a,b \in \{0,1\}^b$ such that $(v,u)\in E \Leftrightarrow \left <a,b \right > =1 $. 
the main property we will want to preserve is connectivity.
That is as a basic requierment we will requier that if $(v,u)\in E \Rightarrow v' \sim u'$. 
For each edge in the original graph we will construct new $18k+2$ vertices labeled by the following names $$ T_i,F_i,T^{xy}_{P,i}  \text{ for }i = 0\dots k-1 \text{and} x,y,P \in \{0,1\}$$. Addionaly we have $T_k,F_k$.
For the following section assume we have some canonical direction for any pair $(v,u)$. 

Alice have the edges $(T_i,T^{{a_i}0}_{0,i}),(F_i,F^{{a_i}0}_{0,i}),(T_i,T^{{a_i}1}_{0,i}),(F_i,F^{{a_i}1}_{0,i})$
while Bob have the edges $(T_{i+1},T^{0{b_i}}_{1,i}),(F_{i+1},F^{0{b_i}}_{1,i}),(T_{i+1},T^{1{b_i}}_{1,i}),(F_{i+1},F^{1{b_i}}_{1,i})$.
Regerdless of Bob and Alice inputs we give them the edges $(T^{xy}_{0,i},T^{xy}_{1,i}) (F^{xy}_{0,i},F^{xy}_{1,i}) \text{ for } x,y \in \{(1,0),(0,0),(0,1)\}$ 
For $x,y = (1,1)$ we add $(T^{11}_{0,i},F^{11}_{1,i})$ and $(F^{11}_{0,i},T^{11}_{1,i})$.
At last we add an edge $(v,F_0)$ and $(u,T_k)$ 

As before we want to show that $\left <a,b \right > =1 \Rightarrow v' \sim u'$, we will prove so by the following claims
\begin{clm}
    $\left < a_{< i},b_{< i} \right > = 1 \Rightarrow v \sim T_{i}$ and $\left < a_{< i},b_{< i} \right > = 0 \Rightarrow v \sim F_{i}$
\end{clm}

\begin{proof}
    We will prove this by induction, the first case where $i=0$ is trivial, as $v$ is clearly connected to $T_0$. 
    It holds that $T_i,F_i$ are connected to $T^{a_{i}b_{i}}_{0,i},T^{a_{i}b_{i}}_{0,i}$ and 
    $T_{i+1},F_{i+1}$ are connected to $T^{a_{i}b_{i}}_{1,i},T^{a_{i}b_{i}}_{1,i}$. If $a_{i} \cdot b_{i} = 0$ then 
    $T^{a_{i}b_{i}}_{0,i},F^{a_{i}b_{i}}_{0,i}$ is connected to $T^{a_{i}b_{i}}_{1,i},F^{a_{i}b_{i}}_{1,i}$ respectively. Therefore
    assume without loss of generality that $\left < a_{< i},b_{< i} \right > = 0$ And therefore $v' \sim F_i$ as 
    $ \left < a_{< i},b_{< i} \right > = \left < a_{< i},b_{< i} \right > + a_i\cdot b_i = \left < a_{< i+1},b_{< i+1} \right > $ 
    the assertion holds as there is a path $F_i \to F^{a_{i}b_{i}}_{0,i} \to F^{a_{i}b_{i}}_{1,i} \to F_{i+1}$
    and respectively there exist a path $T_i \to T_{i+1}$. In the case where $a_i \cdot b_i = 1$ we have that
    $ \left < a_{< i},b_{< i} \right > = \left < a_{< i},b_{< i} \right > + a_i\cdot b_i - 1 = 1 + \left < a_{< i+1},b_{< i+1} \right > $  
    and indeed it holds that we have a path $T_i \to T^{11}_{0,i} \to F^{11}_{1,i} \to F_{i+1}$ and respectively for $F_i \to T_{i+1}$, thus we complete the proof.
\end{proof}

It follows easily that $ \left < a,b \right > \Rightarrow v \sim u$. This claim alone wouldn't be sufficient for our usage (as just always connecting $v$ to $u$ will satisfy this claim as well).
Therefore, we add the following claim 
\begin{clm}
    There is no path inside the gadget such that $v \to p_0 \cdots p_l \to F_i$ as long as $\left < a_{< i},b_{< i} \right > = 1$ (and vice versa for $T_i$). 
\end{clm}

\begin{proof}
    The gadget itself is layered such that reaching $F_i$ is done only via reaching the $i+1$-th or the $i-1$-th layer immediately before.
    We can trace in which vertex the path passed through in each layer, and get the list of vertices.
    On the other hand by Claim "???" we know that $T_i$ is also reachable, and we can create the list of vertices for it as well
    As a consequance we get that in some point the path for $F_i$ and for $T_i$ branch from one another. 
    At this point we must have some vertex $F_j$ (w.l.o.g) that is connected to $T_{j+1} \text{ and } F_{j+1}$ (we chose j+1 w.l.o.g).
    It should be obvies this statment is false as for all four options of $a_j,b_j$, it holds that $F_j$ is only connected to one of $T_{j+1}, F_{j+1}$.
\end{proof}


\subsection{Circle Freeness}

\subsection{Connected Components}

\subsection{???}

\section{Lower Bounds for Property Testing in $R^{dt}$}

\end{document}